\documentclass[a4paper,12pt]{report}

% --------------------------------------------------
%  Packages
% --------------------------------------------------
\usepackage[utf8]{inputenc}  % Support for UTF-8 encoding
\usepackage[english]{babel}  % English language support
\usepackage{amsmath,amssymb} % Math packages
\usepackage{geometry}        % Page geometry
\usepackage{setspace}        % Line spacing
\usepackage{graphicx}        % Insert images
\usepackage{hyperref}        % Hyperlinks in the PDF
\usepackage{float}           % Improved float handling

% Page margins
\geometry{
    left=3cm,
    right=2.5cm,
    top=2.5cm,
    bottom=2.5cm
}

\setstretch{1.3}  % Line spacing

% --------------------------------------------------
%  Begin Document
% --------------------------------------------------
\begin{document}

% --------------------------------------------------
%  Title Page
% --------------------------------------------------
\begin{titlepage}
    \begin{center}
        \vspace*{1.5cm}

        % AUEB Logo
        \includegraphics[width=0.9\textwidth]{./aueb_logo.png}\\[1cm]

        {\Large \textbf{Athens University of Economics and Business}}\\[0.5cm]
        {\large Department of Management Science and Technology}\\[1.5cm]

        {\Huge \textbf{Undergraduate Thesis}}\\[1.2cm]
        {\Large \textbf{Behavioral Profiling of Popular Messaging Apps Using Kernel-Level Tracing with ML Techniques}}\\[2cm]

        \textbf{Student:}\\
        Foivos - Timotheos Proestakis\\
        Student ID: 8210126\\[1.5cm]

        \textbf{Supervisors:}\\
        Prof. Diomidis Spinellis \\
        Dr. Nikolaos Alexopoulos\\[1.5cm]

        \vfill
        \textbf{Submission Date:}\\
        \today
        \vspace*{1cm}
    \end{center}
\end{titlepage}
\clearpage


% --------------------------------------------------
%  Abstract
% --------------------------------------------------
\begin{abstract}
This thesis examines kernel-level tracing techniques to create behavioral profiles of popular messaging applications using Machine Learning. The main goal is to analyze the operational characteristics of such apps and employ ML algorithms to detect patterns regarding security. The study covers topics such as kernel-level data collection, big data processing and analysis, and the design of ML models for behavior identification and classification.
\end{abstract}
\clearpage

% --------------------------------------------------
%  Acknowledgments
% --------------------------------------------------
\chapter*{Acknowledgments}
\addcontentsline{toc}{chapter}{Acknowledgments}

I would like to express my heartfelt gratitude to my supervisors, Prof. Diomidis Spinellis and Dr. Nikolaos Alexopoulos, for their invaluable guidance, insightful feedback, and continuous support throughout the duration of this thesis. Their expertise and encouragement were instrumental in the successful completion of this work.

I would also like to sincerely thank my exceptional fellow students, Vangelis Talos and Giannis Karyotakis, for their contribution, collaboration, and for being true companions in this academic journey.

A special thanks goes to my family, whose unwavering support, both emotional and practical, made this endeavor not only possible but also deeply meaningful. Their presence and encouragement were a constant source of strength.
\clearpage

% --------------------------------------------------
%  Table of Contents
% --------------------------------------------------
\tableofcontents
\clearpage

% --------------------------------------------------
%  Introduction
% --------------------------------------------------
\chapter{Introduction}

Smartphones have become an integral component of modern society, with the number of global users surpassing 5 billion and continuing to grow rapidly \cite{statista2024smartphone}. Among the dominant mobile platforms, Android—an open-source operating system developed by Google—holds a stable global market share of approximately 75\% \cite{statista2021android}. Its open-source nature, flexibility, and widespread adoption have cultivated a vast ecosystem of applications that enhance user productivity and social interaction across various domains.

Among these applications, messaging platforms such as WhatsApp, Telegram, Facebook Messenger, and Signal have gained significant popularity, playing a central role in both personal and professional communication. However, the ubiquitous use of smartphones for such purposes has led to the accumulation of sensitive personal data on user devices, including photos, contact lists, location history, and financial information, thereby raising serious privacy and security concerns \cite{verge2018facebooksms}.

Incidents such as Facebook's unauthorized collection of SMS texts and call logs from Android devices \cite{verge2018facebooksms} underscore the vulnerabilities within existing mobile ecosystems. In response, regulatory frameworks like the General Data Protection Regulation (GDPR) and national laws such as the UK Data Protection Act 2018 aim to enforce principles of transparency, data minimization, and user consent in data processing \cite{gdpr2018, dpa2018}.

Despite these legislative efforts, Android's current permission management system remains insufficient. Users frequently misinterpret the scope and implications of the permissions they grant, inadvertently exposing sensitive data to misuse \cite{feng2020survey, felt2012permissions}.

To address these challenges, it is essential to analyze application behavior—that is, the actual operations performed by an app, both in the foreground and background. Research has shown that discrepancies often exist between user expectations and actual app behavior, with applications executing hidden or unauthorized tasks \cite{uipicker2019, gorla2014checking}. Many detection techniques rely on the assumption that user interface (UI) elements accurately represent application functionality, an assumption that is not always valid \cite{nan2019uipicker}.

Behavioral analysis methods are typically divided into static and dynamic approaches. Static analysis examines application code without execution, identifying known malicious patterns. However, it is susceptible to evasion through obfuscation and polymorphism \cite{arzt2014flowdroid, enck2010taintdroid}. In contrast, dynamic analysis evaluates applications during runtime, monitoring behaviors such as system calls, resource consumption, and network activity \cite{xu2011crowdroid, lindorfer2014andrubis}. Among these, system call analysis is particularly valuable, offering fine-grained visibility into application interactions with hardware and OS-level services \cite{canfora2015syscalls}.

Kernel-level tracing is a powerful form of dynamic analysis, capable of capturing low-level system interactions with high precision. Android is built on a modified Linux kernel that orchestrates resource management and system processes via system calls \cite{love2010linux}. Tools such as \texttt{ftrace} and \texttt{kprobes} enable developers and researchers to trace kernel-level function calls, execution flows, and resource usage \cite{rostedt2023ftrace, kernel2023kprobes}.

\texttt{Ftrace} is a built-in tracing utility within the Linux kernel, optimized for performance and capable of monitoring execution latency and function call sequences. \texttt{Kprobes}, on the other hand, allows for dynamic instrumentation of running kernels, enabling targeted probing of specific code locations during runtime \cite{corbet2015drivers}.

Applying kernel-level tracing to messaging applications, however, introduces unique technical challenges. These apps typically exhibit complex, multi-threaded behavior, frequent background processing, and diverse interactions with system resources. Accurately profiling such behavior requires collecting and interpreting high-volume, high-resolution kernel data \cite{tang2017profiling, kim2016io}.

Despite the growing research interest in Android security and behavioral analysis, existing work has primarily focused on general application profiling or malware detection. Few studies have concentrated specifically on behavioral profiling of messaging apps using kernel-level data \cite{backes2015boxify}. Meanwhile, recent reports concerning the usage of secure messaging apps such as Signal by government and military officials have emphasized the urgent need for transparent, robust behavioral analysis mechanisms \cite{washingtonpost2023signal}.

To address these gaps, this thesis proposes a structured methodology for profiling the behavior of popular messaging applications on Android through kernel-level tracing using \texttt{ftrace} and \texttt{kprobes}. The proposed approach integrates Machine Learning techniques to process and classify behavioral patterns, aiming to enhance security diagnostics, user privacy, and system transparency.

\section{Motivation and Problem Statement}
\paragraph{Motivation}
The motivation behind this research arises from the necessity to bridge existing gaps between user expectations, regulatory compliance, and the actual operational behavior of popular messaging applications. Messaging apps process extensive personal data, creating substantial risks related to privacy violations and security breaches. Recent incidents involving unauthorized data collection by prominent messaging applications, along with revelations about governmental use of supposedly secure messaging platforms, underscore significant concerns regarding transparency and user trust.

\paragraph{Problem Statement}
Current literature lacks comprehensive kernel-level behavioral analyses of messaging applications, leaving critical privacy and security risks inadequately addressed. Thus, this research seeks to systematically explore kernel-level behaviors to enhance transparency, improve user trust, and provide rigorous technical evaluations of messaging applications' privacy implications.

\section{Research Objectives}
The specific research objectives addressed in this thesis are categorized as follows:

\subsection*{Primary Objectives}
\begin{itemize}
\item Record the actual kernel-level behavior of widely used messaging applications.
\item Identify potential violations of the principle of data minimization.
\item Analyze mismatches between granted permissions and real-time resource usage.
\item Detect unauthorized or hidden access to sensitive user data.
\item Compare the behavioral profiles of privacy-focused apps (e.g., Signal) and more commercial alternatives.
\end{itemize}

\subsection*{Analytical and Technical Sub-Objectives}
\begin{itemize}
\item Develop a tracing and profiling framework using ftrace and kprobes.
\item Classify system calls into functional categories (file access, networking, IPC).
\item Monitor transitions between app states (idle, active, background).
\item Collect and analyze kernel-level usage statistics per application.
\item Identify potential indirect data leakage through side-channel patterns.
\item Correlate traced behaviors with declared permissions.
\item Implement a web-based dashboard for behavior visualization.
\end{itemize}

\subsection*{Broader Goals}
\begin{itemize}
\item Enhance transparency in how messaging apps behave at system level.
\item Improve user awareness of hidden behaviors executed in the background.
\item Demonstrate the value of kernel-level tracing for security and privacy evaluation.
\item Provide a structured and reproducible methodology for privacy-respecting behavior analysis.
\end{itemize}

\section{Research Questions}
Based on the motivation and objectives, this thesis aims to address the following research questions:

\vspace{0.5em}
\noindent\fbox{\parbox{\textwidth}{
\textbf{Q1.} What kernel-level operations do popular messaging applications perform during normal usage?
}}

\vspace{0.5em}
\noindent\fbox{\parbox{\textwidth}{
\textbf{Q2.} Are there deviations between the declared permissions of these applications and their actual behavior at runtime?
}}

\vspace{0.5em}
\noindent\fbox{\parbox{\textwidth}{
\textbf{Q3.} Can kernel-level tracing techniques identify unexpected or potentially invasive operations performed without user interaction?
}}

\vspace{0.5em}
\noindent\fbox{\parbox{\textwidth}{
\textbf{Q4.} How does the behavior of privacy-focused apps compare to that of commercial messaging platforms at the kernel level?
}}

\vspace{0.5em}
\noindent\fbox{\parbox{\textwidth}{
\textbf{Q5.} What kind of patterns in system calls can be used to characterize privacy-relevant behavior?
}}


\section{Limitations}
\begin{itemize}
    \item \textbf{Platform Scope}: Analysis restricted to Android 10+ due to kernel API dependencies.
    \item \textbf{Dynamic Analysis Constraints}: Real-world noise (e.g., background services) may affect system call traces.
    \item \textbf{App Selection Bias}: Focus on top-tier apps (WhatsApp, Signal, Telegram) may omit niche platforms.
\end{itemize}

\section{Contributions of this Thesis}

\section{Thesis Outline}
This thesis is organized into the following chapters:

\begin{itemize}
\item \textbf{Chapter 1 – Introduction:} Provides background context, outlines the motivation and objectives, presents the research questions, contributions, and a high-level overview of the thesis structure.
\item \textbf{Chapter 2 – Related Work and Technical Background:} Reviews existing literature on Android architecture, messaging app privacy implications, static and dynamic analysis techniques, system call tracing, and identifies key research gaps.
\item \textbf{Chapter 3 – Methodology and System Design:} Describes the research design, experimental setup, data collection using kernel-level tracing, and the analysis framework.
\item \textbf{Chapter 4 – Results:} Presents the observed behavioral patterns, differences among messaging apps, and key findings related to privacy-relevant behaviors.
\item \textbf{Chapter 5 – Discussion:} Interprets the results in light of the research questions, discusses limitations of the study, and suggests potential improvements.
\item \textbf{Chapter 6 – Conclusions:} Summarizes key contributions, highlights findings, and suggests directions for future research.
\item \textbf{Appendix A – Additional Data Tables:} Includes supplementary statistical data and traces.
\item \textbf{Appendix B – Code:} Provides relevant shell scripts, Python tools, and configuration details used in the implementation.
\end{itemize}


% --------------------------------------------------
%  Related Work and Technical Background
% --------------------------------------------------
\chapter{Related Work and Technical Background}
This thesis is part of a broader research effort investigating the security and behavioral analysis of Android applications at the kernel level. While the present work focuses on behavioral profiling for privacy analysis, other components of the research include detection mechanisms using machine learning, portability of tracing techniques across devices, and offset-agnostic instrumentation. These aspects are discussed in parallel efforts by the research team, but are outside the scope of this thesis.

This chapter provides a detailed overview of related work and technical background necessary for understanding the methodology and objectives of this thesis. First, it presents the architecture of the Android operating system, focusing particularly on the Linux-based kernel and how applications interact with it. Next, it discusses the behavior and privacy concerns related to messaging applications, highlighting known issues and relevant technical aspects. Furthermore, it outlines the advantages and limitations of static and dynamic analysis techniques and explores the role of system calls in behavior profiling. Finally, it reviews kernel-level tracing tools and techniques, and identifies gaps in existing research where this thesis contributes.
\section{Android Architecture and Kernel-Level Access}

\subsection{Android Software Stack Overview}
Android is a layered, open-source mobile operating system built on top of a customized version of the Linux kernel. Its architecture is designed to be modular and extensible, supporting a wide range of hardware while enforcing clear boundaries between components. The Android software stack consists of four major layers: the Application Layer, the Java API Framework (commonly referred to as the Application Framework), the Hardware Abstraction Layer (HAL), and the Linux Kernel.

The Application Layer hosts both system and user-installed applications. These applications interact with the system via APIs exposed by the Android Framework. The Java API Framework provides access to core system services such as activity management, resource handling, content providers, and telephony. Services like \texttt{ActivityManager}, \texttt{WindowManager}, and \texttt{PackageManager} facilitate the lifecycle management and orchestration of application behavior.

Beneath the framework lies the Android Runtime (ART), which executes application bytecode and optimizes it using ahead-of-time (AOT), just-in-time (JIT), or interpretation modes. Alongside ART are native libraries written in C/C++, including performance-critical components such as WebView, OpenSSL, and the Bionic libc. The Java Native Interface (JNI) allows managed Java/Kotlin code to call into these native libraries.

The HAL acts as a bridge between the Android Framework and the hardware drivers residing in the kernel. It defines standard interfaces that vendors implement to support various hardware components like audio, camera, sensors, and graphics. Since Android 10, Google introduced the Generic Kernel Image (GKI), which aims to further separate the vendor-specific hardware implementations from the core Linux kernel by introducing a stable kernel interface. This allows devices from different manufacturers to share a common kernel base while maintaining vendor-specific modules separately, simplifying updates and enhancing portability.

At the lowest level, the Linux kernel provides essential operating system services such as process scheduling, memory management, networking, and security enforcement. Android extends the kernel with additional features including the Binder IPC driver, ashmem (anonymous shared memory), and wakelocks to manage power usage. This kernel foundation ensures that resource access is isolated and controlled across all system layers.

\textbf{Figure 1:} Updated Diagram of Android Software Stack (source: Android Developers Guide~\cite{androidplatformdoc}).


\subsection{Application Layer and Process Lifecycle}
The Application Framework manages essential system services such as the ActivityManagerService, PackageManagerService, and NotificationManager. Applications interact via Binder IPC, enabling efficient inter-process communication. The process lifecycle (e.g., foreground, background, idle states) is regulated by these system services.
\textbf{Figure 2:} Application Lifecycle State Machine.

\subsection{2.1.3 Android Runtime, Native Layer, and JNI}
Android Runtime (ART) manages app execution, compiling bytecode into optimized native instructions. The native layer includes libraries such as Bionic libc, which provide a lightweight and efficient runtime environment. JNI bridges Java code and native C/C++ libraries, facilitating performance-critical operations.

\subsection{2.1.4 Linux Kernel Fundamentals in Android}
The Linux kernel performs critical functions including memory management, scheduling, power management, and hardware interactions. Key Android-specific features include wakelocks for power management, ashmem for shared memory, and binder for IPC. Kernel components ensure isolation between user and kernel spaces.

\subsection{2.1.5 System Calls and Kernel Interaction}
Applications communicate with the kernel through system calls, invoking operations for file management (open, read, write), networking (socket operations), and IPC. These system calls serve as a fundamental interface, enabling controlled resource access.

\subsection{2.1.6 Android Security Model and Isolation Mechanisms}
Android employs sandboxing via UID/GID for each application, filesystem isolation under \texttt{/data/data/}, and permission models enforced at runtime. SELinux provides mandatory access control policies, restricting system calls through mechanisms such as seccomp, enhancing the overall security posture.

\subsection{2.1.7 Kernel Tracing and Monitoring Interfaces}
Kernel-level tracing provides powerful insights into application behaviors. Tools like ftrace, available through \texttt{/sys/kernel/debug/tracing/}, monitor kernel events and system calls. Kprobes allow dynamic instrumentation by inserting probes at runtime. Advanced techniques such as tracepoints and uprobes offer additional granularity, with user-space alternatives (strace) providing limited visibility. Emerging technologies like eBPF further extend tracing capabilities, though practical use in production Android systems remains constrained.

\subsection*{Figures and Diagrams}
\begin{itemize}
\item Figure 1: Android Software Stack
\item Figure 2: Application Lifecycle
\item Figure 3: Kernel and User-Space Separation
\item Figure 4: Binder IPC Mechanism
\item Figure 5: SELinux and Access Control Flow
\item Figure 6: System Call Interaction Flow
\item Figure 7: Kernel Tracing Pipeline
\end{itemize}

Relevant sources and additional references include official Android documentation, Linux kernel manuals, and peer-reviewed papers on Android system architecture and security.
\section{Messaging Apps: Characteristics  Privacy Implications}


\section{Static vs. Dynamic Analysis}

\section{System Call Analysis and Kernel Tracing}

\section{Machine Learning for Behavior Profiling}

\section{Related Research  Gaps in Literature}

% --------------------------------------------------
%  Research Methodology
% --------------------------------------------------
\chapter{Methodology and System Design}

\section{Research Design}

\section{Experimental Setup}
A discussion of experimental settings, including how experiments were conducted and what evaluation metrics (e.g., accuracy, precision, recall, F1-score) were used.



\section{Data Collection and Preparation}
A detailed description of how kernel-level data is collected and the preprocessing steps taken to ensure suitability for ML algorithms.

\section{Feature Extraction from System Calls}

\section{Machine Learning Models and Tools}
An overview of the ML algorithms (e.g., Random Forest, SVM, Neural Networks) and the software tools (e.g., Python, scikit-learn) employed in the study.




% --------------------------------------------------
%  Results
% --------------------------------------------------
\chapter{Results}
\section{Behavioral Patterns Observed}

\section{Model Performance}

\section{Comparison Between Messaging Apps}

\section{Classification or Pattern Recognition Outcomes}
Presentation of evaluation tables, charts, and analysis derived from the ML algorithms.

\section{Comparisons and Interpretations}
Comparison of different models or configurations, with emphasis on interpreting discrepancies and assessing each model’s performance.

\section{ Visualizations}

% --------------------------------------------------
%  Discussion
% --------------------------------------------------
\chapter{Discussion}

\section{Interpretation of Results}

\section{Limitations of the Study}

\section{Opportunities for Improvement}


A detailed discussion of how the findings relate to the initial research objectives and the broader literature. The contribution and limitations of this study are highlighted.

% --------------------------------------------------
%  Conclusions
% --------------------------------------------------
\chapter{Conclusions}

\section{Key Findings}
A summary of the main results and how they address the initial research questions.

\section{Future Research Directions}
Suggestions for expanding this research, including improvements or new avenues for study.

% --------------------------------------------------
%  References
% --------------------------------------------------
\begin{thebibliography}{99}
    \bibitem{statista2024smartphone} Statista, \textit{Number of smartphone users worldwide from 2014 to 2029}, 2024. Available: \url{https://www.statista.com/forecasts/1143723/smartphone-users-in-the-world}

    \bibitem{statista2021android} F. Laricchia, "Mobile operating systems’ market share worldwide from January 2012 to July 2020," Statista, 2021. \url{https://www.statista.com/statistics/272698/global-market-share-held-by-mobile-operating-systems-since-2009/}

    \bibitem{verge2018facebooksms} The Verge, "Facebook has been collecting call history and SMS data from Android devices," 2018. \url{https://www.theverge.com/2018/3/25/17160944/facebook-call-history-sms-data-collection-android}

    \bibitem{gdpr2018} GDPR.EU, \textit{General Data Protection Regulation}, 2018. \url{https://gdpr.eu/}

    \bibitem{dpa2018} UK Government, \textit{Data Protection Act 2018}, \url{https://www.legislation.gov.uk/ukpga/2018/12/contents/enacted}

    \bibitem{feng2020survey} Z. Feng et al., "A Survey on Security and Privacy Issues in Android," IEEE Communications Surveys \& Tutorials, vol. 22, no. 4, pp. 2445-2472, 2020.

    \bibitem{felt2012permissions} A. Felt et al., "Android Permissions: User Attention, Comprehension, and Behavior," SOUPS, 2012.

    \bibitem{gorla2014checking} A. Gorla et al., "Checking App Behavior Against App Descriptions," ICSE, 2014.

    \bibitem{nan2019uipicker} Y. Nan et al., "UIPicker: User-Input Privacy Identification in Mobile Applications," IEEE TSE, 2019.

    \bibitem{arzt2014flowdroid} S. Arzt et al., "FlowDroid: Precise Context, Flow, Field, Object-sensitive and Lifecycle-aware Taint Analysis for Android Apps," PLDI, 2014.

    \bibitem{enck2010taintdroid} W. Enck et al., "TaintDroid: An Information-Flow Tracking System for Realtime Privacy Monitoring on Smartphones," OSDI, 2010.

    \bibitem{xu2011crowdroid} Z. Xu et al., "Crowdroid: Behavior-Based Malware Detection System for Android," SPSM, 2011.

    \bibitem{lindorfer2014andrubis} M. Lindorfer et al., "ANDRUBIS - 1,000,000 Apps Later: A View on Current Android Malware Behaviors," BADGERS, 2014.

    \bibitem{canfora2015syscalls} G. Canfora et al., "Detecting Android Malware Using Sequences of System Calls," IEEE TSE, 2015.

    \bibitem{love2010linux} R. Love, \textit{Linux Kernel Development}, Addison-Wesley, 2010.

    \bibitem{rostedt2023ftrace} S. Rostedt, "Ftrace: Function Tracer," Linux Kernel Documentation, 2023.

    \bibitem{kernel2023kprobes} Linux Kernel Organization, "Kernel Probes (kprobes)," Linux Kernel Documentation, 2023.

    \bibitem{corbet2015drivers} J. Corbet, G. Kroah-Hartman, A. Rubini, \textit{Linux Device Drivers}, 4th ed., O'Reilly Media, 2015.

    \bibitem{tang2017profiling} J. Tang et al., "Profiling Android Applications via Kernel Tracing," IEEE TMC, 2017.

    \bibitem{kim2016io} J. Kim et al., "Understanding I/O Behavior in Android Applications through Kernel Tracing," ACM MobiSys, 2016.

    \bibitem{backes2015boxify} M. Backes et al., "Boxify: Full-fledged App Sandboxing for Stock Android," USENIX Security, 2015.

    \bibitem{washingtonpost2023signal} The Washington Post, "Pentagon officials used Signal messaging app, raising security concerns," March 2023.
    \end{thebibliography}
\clearpage

% --------------------------------------------------
%  Appendices
% --------------------------------------------------
\appendix

\chapter{Appendix A: Additional Data Tables}
Any further data tables, graphics, or supplementary material.

\chapter{Appendix B: Code}
Source code or additional scripts too extensive to include in the main chapters.

\end{document}
